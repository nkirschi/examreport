% step 1: define exam characteristics
\newcommand{\authors}{Wazlaf Wttlbrmft} % authors of this transcript
\newcommand{\lecture}{Great Lecture II} % lecture title
\newcommand{\semester}{WS 2021/22} % semester and year
\newcommand{\chair}{Chair of Applied Joy} % responsible chair
\newcommand{\lecturer}{Prof. Dr. Nathan Expla} % lecturer
\newcommand{\tutors}{Dr. Alan Tutor} % tutors, if applicable
\newcommand{\duration}{1337} % exam duration in minutes
\newcommand{\maxpoints}{42} % total number of attainable points
\newcommand{\aids}{Pen and brain} % permitted aids, if applicable

% step 2: load template with desired language (english or german)
\documentclass[english]{examreport}

% step 3: record the exam questions
\begin{document}
  \begin{examreport}
    \begin{task}[title = Optional title, points = 42]
      Complete this task.
      \begin{enumerate}[(a)]
        \item Prove that there exists a field where \(1 + 3 + 3 + 7 = 0\).
        \item Give the answer to life, the universe, and everything.
      \end{enumerate}
    \end{task}
  \end{examreport}
\end{document}
